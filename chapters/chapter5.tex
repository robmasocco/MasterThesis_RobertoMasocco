% Chapter 5
% Roberto Masocco <robmasocco@gmail.com>
% October 8, 2021

\chapter[Conclusioni]{Conclusioni}
\label{chap:Chapter5} 
\doublespacing
\fontsize{12}{12}\selectfont
\indent In questo lavoro è stata presentata una nuova serie di soluzioni hardware e software finalizzate allo sviluppo ed al deployment di architetture di controllo per sistemi autonomi complessi. Nel Capitolo \ref{chap:Chapter2} è stato presentato il concetto di middleware come strato di astrazione intermedio tra l'hardware ed il software applicativo volto a semplificare alcune problematiche classiche della programmazione di apparati autonomi. Particolare attenzione è stata rivolta a ROS 2 come soluzione di nuova concezione basata su DDS e completamente open-source. Nel Capitolo \ref{chap:Chapter3} è stato presentato il caso di un drone autonomo per volo automatico, capace di localizzarsi in un ambiente privo di segnale GNSS e compiervi una missione basata su esplorazione dell'ambiente ed atterraggi multipli in zone predefinite. L'architettura di controllo è stata interamente costruita impiegando il middleware ROS 2 a supporto dello sviluppo del software di acquisizione dati, controllo e supervisione, nonché per gestire le comunicazioni tra i vari moduli e la postazione dell'operatore. Si è proceduto dunque ad un'accurata descrizione della sua implementazione e realizzazione e ad un commento dei risultati ottenuti. Nel Capitolo \ref{chap:Chapter4} è stato poi presentato un ulteriore caso di studio relativo all'implementazione su tale architettura di un controllore veloce per l'esecuzione di atterraggi di precisione. Sono stati esposti nei dettagli i limiti operativi ed applicativi di una tale soluzione, e sono state descritte le varie fasi del processo di identificazione del sistema da controllare. Successivamente si è discussa l'implementazione del controllore ed i risultati ottenuti nelle sue varie prove sperimentali.

\newsection{Sviluppi futuri}
\indent Il presente lavoro dimostra quali siano la versatilità e l'ampio ventaglio di possibilità aperte dalle soluzioni proposte. Come si è detto nel Capitolo \ref{chap:Chapter1}, la direzione intrapresa dai recenti sviluppi tecnologici indica come le problematiche risolte da questi strumenti siano sempre più attuali, ed i risultati ottenuti ne provano l'efficacia e l'efficienza. Dal punto di vista dei middleware, ed in particolare dei DDS, il sempre più vicino deployment delle reti 5G aprirà alla possibilità di costruire architetture di controllo distribuite ad elevata capillarità, alta efficienza e bassa latenza, il caso d'uso per eccellenza di questo tipo di framework. Riguardo ROS 2 invece, nonostante l'ampia presentazione che se ne è fatta nel Capitolo \ref{chap:Chapter2} ed i numerosi rimandi alle parti tralasciate per brevità, è stata solo scalfita la superficie. Per favorirne l'integrazione con sistemi ampiamente diversificati, sono attualmente in sviluppo versioni ridotte di questo middleware che hanno come target sistemi embedded e a risorse limitate, particolarmente indicate per implementare controllori come quello descritto nel Capitolo \ref{chap:Chapter4}, il quale nonostante tutto ha funzionato a dovere su un sistema general-purpose su cui era in esecuzione anche il resto dell'architettura più il simulatore. Queste soluzioni sarebbero da accoppiare con altre, sempre in sviluppo ma già aperte alla sperimentazione, volte a rendere i DDS compatibili con link di comunicazione diversi da quelli di Rete, come bus seriali o paralleli, in modo simile a quanto visto nel Capitolo \ref{chap:Chapter3}. Un altro aspetto molto interessante ma che non è stato toccato in questo lavoro è poi quello della containerizzazione: ROS 2 offre già, mediante il proprio build system, la possibilità di compilare nodi e package come librerie condivise a caricamento dinamico. Ciò rende possibile attivare e disattivare i nodi a piacimento in modo totalmente configurabile, migliorando il carico e le prestazioni del sistema ma anche rendendo più efficiente l'implementazione in generale, dato che tutti i moduli attivi in un dato istante risiederebbero nello stesso \emph{address space}. Sarebbe anche possibile gestire più semplicemente situazioni eccezionali come i guasti, nonché eseguire il testing dei singoli moduli e della loro integrazione, evitando di dover portare offline l'intera architettura per agire su anche solo un singolo modulo. Infine, un altro ambito da esplorare è quello dell'implementazione di schedulazioni hard real-time, nei limiti consentiti dall'impiego di un kernel Linux PREEMPT su un sistema quasi general-purpose, che coinvolgano i processi che fanno uso del middleware.
Riguardo lo specifico ambito dei droni, essere riusciti a metterne in volo uno così complesso e totalmente autonomo dimostra come molto altro si possa fare usando il framework costruito. Dal punto di vista della sensoristica di bordo si può integrare il sistema di localizzazione e mapping ORB\_SLAM2 con altre misure, ottenute mediante algoritmi diversi e fuse secondo opportuni algoritmi di stima robustificati. È poi possibile lavorare sui livelli più bassi del sistema di controllo veloce, essendo anche questo totalmente aperto, per implementare leggi di controllo più sofisticate e raffinate, conseguendo nuove e migliori performance di volo. La disponibilità di un'architettura di controllo naturalmente distribuita e così facile da impiegare rende infine possibile costruire flotte di droni cooperativi, che realizzino autonomamente ed in gruppo task simili a quelli discussi. Infine, nel Capitolo \ref{chap:Chapter4} è stato presentato un caso d'applicazione di un sistema di controllo switching, mostrando come le performance sotto la sua azione migliorassero sensibilmente rispetto ad una soluzione standard. È questo un ambito che merita uno studio approfondito, corredato da prove sperimentali ed implementazioni reali rese semplicemente possibili da architetture come quella proposta. Relativamente al caso di specie si può studiare l'impiego di un controllore switching in feedback dallo stato, invece che dalla misura dell'errore, che preveda più d'una commutazione.
